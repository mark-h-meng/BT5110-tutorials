\section*{Question 1}

\begin{frame}{Question 1 (a-b)}
    (a) Download the following files from Luminus ``Files > Cases > Book Exchange''.
			\begin{itemize}
				\item[] \texttt{NUNStASchema.sql},
				\item[] \texttt{NUNStAStudent.sql},
				\item[] \texttt{NUNStABook.sql},
				\item[] \texttt{NUNStACopy.sql},
				\item[] \texttt{NUNStALoan.sql}, and \texttt{NUNStAClean.sql}.
			\end{itemize}
			
    (b) Read the SQL files. What are they doing? 
    
    (c) Use the files to create and populate a database (they may need some bug fixing).
\end{frame}

\begin{frame}{Answer 1 (b)}
	The first file to be run is \texttt{NUNStASchema.sql}. It creates the different tables. The referential integrity constraints (\texttt{FOREIGN KEY}) impose that the table \texttt{copy} and the table \texttt{loan} are created after the tables \texttt{student} and \texttt{book} and in that order.
	
	The table \texttt{student} is populated using \texttt{NUNStAStudent.sql}. 
	
	The table \texttt{book} is populated using \texttt{NUNStABook.sql}. 
	
	\textbf{\textit{These two tables can be populated in any order.}} 
\end{frame}

\begin{frame}{Answer 1 (b) Cont.}
	The table \texttt{copy} is populated using \texttt{NUNStACopy.sql}. 
	
	The table \texttt{loan} is populated using \texttt{NUNStALoan.sql}. 
	
	\textit{\textbf{The table \texttt{copy} and the table \texttt{loan} can only be populated after the tables \texttt{student} and \texttt{book} are populated and in that order because of the referential integrity constraints (\texttt{FOREIGN KEY}).} }
	
	The referential integrity constraints (\texttt{FOREIGN KEY}) impose that the table \texttt{loan} and the table \texttt{copy} are deleted before the tables \texttt{student} and \texttt{book} and in that order in \texttt{NUNStAClean.sql} matters.
\end{frame}

\begin{frame}{Answer 1 (c)}

    There is a bug in \texttt{NUNStASchema.sql}, as the populating order of table \texttt{loan} (line 29-39) and \texttt{copy} (line 41-47) are wrong. 
    
    You can fix it by \textbf{\textit{swapping}} these two code sections.
    
    Then execute all SQL files except \texttt{NUNStAClean.sql} through PgAdmin.
\end{frame}

\section*{Question 2}

\begin{frame}{Question 2 (a)}
    Insert the following new book.
	
	\begin{lstlisting}
		INSERT INTO book VALUES (
		'An Introduction to Database Systems',
		'paperback' , 
		640 , 
		'English' , 
		'C. J. Date' , 
		'Pearson',
		'2003-01-01' , 
		'0321197844' , 
		'978-0321197849');
    \end{lstlisting}
\end{frame}

\begin{frame}{Answer 2 (a)}
    Notice the implicit order of the fields.

	You can check that the insertion was effective with the following query.
	\begin{lstlisting}
		SELECT * FROM book;
    \end{lstlisting}

\end{frame}

\begin{frame}{Question 2 (b)}
Insert the same book with a different \texttt{ISBN13}, for instance \texttt{'978-0201385908'}.
\end{frame}

\begin{frame}{Answer 2 (b)}
\begin{lstlisting}
		INSERT INTO book VALUES (
		'An Introduction to Database Systems', 
		'paperback', 
		640,
		'English',
		'C.J. Date', 
		'Pearson', 
		'2003-01-01', 
		'0321197844',  
		'978-0201385908');
\end{lstlisting}
	
	The command yields an error  because \texttt{ISBN10} must be unique. PostgreSQL returns the following error message.
	
	\begin{verbatim}
		ERROR:  duplicate key value violates unique constraint "book_isbn10_key"
		DETAIL:  Key (isbn10)=(0321197844) already exists.
		SQL state: 23505
	\end{verbatim}
	
\end{frame}

\begin{frame}{Answer 2 (b) Cont.}
    \textcolor{red}{All messages emitted by the PostgreSQL server are assigned five-character error codes that follow the SQL standard's conventions for ``SQLSTATE'' codes. Applications that need to know which error condition has occurred should usually test the error code, rather than looking at the textual error message.}
		See \url{www.postgresql.org/docs/13/errcodes-appendix.html}
\end{frame}

\begin{frame}{Question 2 (c)}
Insert the same book  with the original \texttt{ISBN13} but with a different \texttt{ISBN10}, for instance \texttt{'0201385902'}.
\end{frame}


\begin{frame}{Answer 2 (c)}
\begin{lstlisting}
		INSERT INTO book VALUES (
		'An Introduction to Database Systems', 
		'hardcover',
		938,
		'English',
		'C.J. Date',
		'Addison Wesley Longman',
		'2000-01-01',
		'0201385902',
		'978-0321197849');
\end{lstlisting}
	
	The command yields an error  because  \texttt{ISBN13} is a primary key and therefore unique. PostgreSQL returns the following error message.
	
	\begin{verbatim}
		ERROR:  duplicate key value violates unique constraint "book_pkey"
		DETAIL:  Key (isbn13)=(978-0321197849) already exists.
		SQL state: 23505
	\end{verbatim}
\end{frame}


\begin{frame}[fragile]{Second Slide Title} 
	
	\begin{itemize}
		\item Avertir Drupal
	\end{itemize}
  \begin{lstlisting}
  	ERROR:  relation "copy" does not exist
    SQL state: 42P01
  \end{lstlisting}
  
  2-b
  \begin{lstlisting}
  ERROR:  duplicate key value violates unique constraint "book_isbn10_key"
DETAIL:  Key (isbn10)=(0321197844) already exists.
SQL state: 23505
  \end{lstlisting}
  
  2-c
  \begin{lstlisting}
  ERROR:  duplicate key value violates unique constraint "book_pkey"
DETAIL:  Key (isbn13)=(978-0321197849) already exists.
SQL state: 23505
  \end{lstlisting}
  
  2-d
  \begin{lstlisting}
  ERROR:  duplicate key value violates unique constraint "student_pkey"
DETAIL:  Key (email)=(tikki@gmail.com) already exists.
SQL state: 23505
  \end{lstlisting}
  
  2-e (extra)
  \begin{lstlisting}
  ERROR:  null value in column "email" of relation "student" violates not-null constraint
DETAIL:  Failing row contains (RIKKI TAVI , null, 2010-01-01, School of Computing , CS, null).
SQL state: 23502
  \end{lstlisting}
  
  2-g
  \begin{lstlisting}
  ERROR:  update or delete on table "student" violates foreign key constraint "loan_borrower_fkey" on table "loan"
DETAIL:  Key (email)=(xiexin2011@gmail.com) is still referenced from table "loan".
SQL state: 23503
  \end{lstlisting}
  
  3-b (case 1: IMMEDIATE)
  \begin{lstlisting}
  ERROR:  current transaction is aborted, commands ignored until end of transaction block
SQL state: 25P02
  \end{lstlisting}
  
  3-b (case 2: DEFFERED)
  \begin{lstlisting}
  ERROR:  syntax error at or near "DEFERRED"
LINE 1: DEFERRED
        ^
SQL state: 42601
Character: 1
  \end{lstlisting}
  
  2-g
  \begin{lstlisting}
  
  \end{lstlisting}
\end{frame}
