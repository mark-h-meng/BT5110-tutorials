\title{BT5110 Data Management and Warehousing}

\subtitle{Tutorial 5: Normalisation}

\author{Mark Meng Huasong}

\institute[National University of Singapore] % (optional, but mostly needed)
{
	School of Computing\\
	National University of Singapore
}

\titlegraphic{
	\includegraphics[width=2cm]{nus-logo}
}

\date{4 - 8 Oct, 2021}

\begin{frame}
	\titlepage
	\begin{tcolorbox}
		\begin{center}
			{\scriptsize \textcolor{red}{All the materials within presentation slides are protected by copyrights.\\
					It is forbidden by NUS to upload these materials to the Internet.}}
		\end{center}
	\end{tcolorbox}
\end{frame}

\begin{frame}[fragile]{Quick Recap: End of Last Tutorial}
	What we have done in the previous class:\\\vspace{5pt}
	(1) Entity-relationship diagram design. \\
	(2) Transforming ER diagram into SQL data description language (DDL)\\\vspace{5pt}
	\begin{figure}
		\includegraphics[width=0.75\textwidth, trim=0 0 0 0, clip]{t5/images/steps.png}
	\end{figure}
\end{frame}

\section*{Question 1 Functional Dependencies}

\begin{frame}[fragile]{Question 1}
Your company, Apasaja Private Limited, is commissioned by an online company offering several services to design the relational schema the management of their users' profiles. A service is fully described and identified by its \textbf{name}. Each user can register to one or more services. A user is uniquely identified by her \textbf{email} as well as by her \textbf{mobile number}. Each user has a \textbf{postal address} and a \textbf{country} of residence. The postal address, however, unambiguously identifies the coutry in which it is located. There can be several users with the same address.\vspace{5pt}

However, we are only given an abstract schema for this application. \vspace{5pt}

Consider the relations $R=\{A, B, C, D, E\}$ with the set of functional dependencies $\Sigma=\{\{A\} \rightarrow \{A, B, C\}, \{A, B\} \rightarrow \{A\}, \{B, C\} \rightarrow \{A, D\}, \{B\} \rightarrow \{A, B\}, \{C\} \rightarrow \{D\}\}$.\vspace{5pt}
\end{frame}

\begin{frame}[fragile]{Question 1 (a) Mapping}
(a) From the functional dependencies in $\Sigma$ and the text description of the application, can you figure out a mapping of the attributes and the letters? \vspace{15pt}

\textbf{Solution}: $A$ and $B$ are mobile number and email, or the other way around.\vspace{5pt}

$C$ is the address.\vspace{5pt}

$D$ is the country.\vspace{5pt}

$E$ is the name of a service.\vspace{5pt}
\end{frame}

\begin{frame}[fragile]{Question 1 (b) Attribute Closure}
(b) Compute the attribute closures of the attributes in $R$ with $\Sigma$ in order to find the candidate keys of $R$ with $\Sigma$. \vspace{15pt}

\textbf{Solution}: \vspace{5pt}

\begin{columns}[t]
	\column{0.45\textwidth}
	\textcolor{gray}{\scriptsize \textit{(Let's start with single attribute first)}}\\
	$\{A\}^{+}= \{A, B, C, D\}$\\	
	$\{B\}^{+}= \{A, B, C, D\}$\\	
	$\{C\}^{+}= \{C, D\}$\\
	$\{D\}^{+}= \{D\}$\\
	$\{E\}^{+}= \{E\}$\\ \vspace{5pt}
	\textcolor{gray}{\textit{\scriptsize (Two attributes' combination)}}\\
	$\{A, B\}^{+}= \{A, B, C, D\}$\\
	$\{A, C\}^{+}= \{A, B, C, D\}$\\
	$\{A, D\}^{+}= \{A, B, C, D\}$
	\column{0.45\textwidth}	
	$\underline{\{A, E\}^{+}}= \{A, B, C, D, E\}$\\
	$\{B, C\}^{+}= \{A, B, C, D\}$\\
	$\{B, D\}^{+}= \{A, B, C, D\}$\\
	$\underline{\{B, E\}^{+}}= \{A, B, C, D, E\}$\\
	$\{C, D\}^{+}= \{C, D\}$\\
	$\{C, E\}^{+}= \{C, D, E\}$\\
	$\{D, E\}^{+}= \{D, E\}$\\ \vspace{5pt}
	Other attribute closures need not be computed.
\end{columns}
\end{frame}

\begin{frame}[fragile]{Question 1 (c-d)}
	(c) Find the candidate keys of $R$ with $\Sigma$. \\ \vspace{5pt}
	\textbf{Solution}: The candidate keys are $\{A, E\}$ and $\{B, E\}$. \\\vspace{35pt}

	(d) Find the prime attributes.\\ \vspace{5pt}
	\textbf{Solution}: The prime attributes are $A$, $B$ and $E$.
\end{frame}

\begin{frame}[fragile]{Question 1 Cont. \textcolor{red}{(Extra)}}
\begin{figure}
	\includegraphics[width=0.85\textwidth, trim=0 0 0 0, clip]{t5/images/keys.png}
\end{figure}
\end{frame}

\section*{Question 2 Minimal Cover}

\begin{frame}[fragile]{Question 2 (a) Minimal Cover.}
	(a) Compute a minimal cover of $R$ with $\Sigma$.\vspace{10pt}

	\textbf{Solution}: \\\vspace{3pt}
	We start from $\Sigma$: \\\vspace{5pt}
	$\{A\} \rightarrow \{A, B, C\}$\\
	$\{A, B\} \rightarrow \{A\}$\\
	$\{B, C\} \rightarrow \{A, D\}$\\
	$\{B\} \rightarrow \{A, B\}$\\
	$\{C\} \rightarrow \{D\}$\\\vspace{5pt}
	Step 1, we simplify the right-hand sides:\\\vspace{3pt}
	\begin{columns}[t]
	\column{0.45\textwidth}
	$\{A\} \rightarrow \{A\}$\\
	$\{A\} \rightarrow \{B\}$\\
	$\{A\} \rightarrow \{C\}$\\
	$\{A, B\} \rightarrow \{A\}$\\
	$\{B, C\} \rightarrow \{A\}$
	\column{0.45\textwidth}
	$\{B, C\} \rightarrow \{D\}$\\
	$\{B\} \rightarrow \{A\}$\\
	$\{B\} \rightarrow \{B\}$\\
	$\{C\} \rightarrow \{D\}$
	\end{columns}
\end{frame}

\begin{frame}[fragile]{Question 2 (a) Cont.}
	Step 2, we simplify the left-hand sides:\\\vspace{3pt}
	$\{A\} \rightarrow \{A\}$.\\	
	$\{A\} \rightarrow \{B\}$.\\	
	$\{A\} \rightarrow \{C\}$.\\	
	$\{A,\cancel{B}\} \rightarrow \{A\}$ because $\{A\} \rightarrow \{A\}$.\\	
	$\{B, \cancel{C}\} \rightarrow \{A\}$ because $\{B\} \rightarrow \{A\}$.\\	
	$\{B, \cancel{C}\} \rightarrow \{D\}$ because $\{B\} \rightarrow \{D\}$ \\\vspace{3pt}
	(we could also do $\{\cancel{B}, C\} \rightarrow \{D\}$ because $\{C\} \rightarrow \{D\}$). Note that we know that $\{B\} \rightarrow \{D\}$ because $\{B\}^{+}= \{A, B, C, D\}$.\\\vspace{3pt}
	$\{B\} \rightarrow \{A\}$.\\
	$\{B\} \rightarrow \{B\}$.\\
	$\{C\} \rightarrow \{D\}$.
\end{frame}

\begin{frame}[fragile]{Question 2 (a) Cont.}
	Step 3, we simplify the set:\\\vspace{3pt}
	$\cancel{\{A\} \rightarrow \{A\}}$ because it is trivial.\\	
	$\{A\} \rightarrow \{B\}$.\\	
	$\{A\} \rightarrow \{C\}$.\\	
	$\{B\} \rightarrow \{A\}$.\\	
	$\cancel{\{B\} \rightarrow \{D\}}$ because it can be derived from the others.\\	
	$\cancel{\{B\} \rightarrow \{B\}}$ $\cancel{\{A\} \rightarrow \{A\}}$.\\	
	$\{C\} \rightarrow \{D\}$.\\\vspace{5pt}
	
	\textbf{The result is:} \\\vspace{3pt}
	$\{A\} \rightarrow \{B\}$\\	
	$\{A\} \rightarrow \{C\}$\\	
	$\{B\} \rightarrow \{A\}$\\
	$\{C\} \rightarrow \{D\}$
\end{frame}

\begin{frame}[fragile]{Question 2 (a) Cont.}
	\begin{alertblock}{Notice}
	Note that there can be other minimal covers that the algorithm can compute by considering the functional dependencies in a different order at each step of the algorithm. This is not the case in the example.
	\end{alertblock}\vspace{5pt}

	However, there is a minimal cover that the algorithm cannot compute:\\\vspace{5pt}
	$\{A\} \rightarrow \{B\}$\\
	$\{B\} \rightarrow \{A\}$\\
	$\{B\} \rightarrow \{C\}$\\
	$\{C\} \rightarrow \{D\}$\\\vspace{10pt}	
	If the algorithm starts from $\Sigma^{+}$, then it can find all minimal covers.
\end{frame}

\begin{frame}[fragile]{Question 2 (b)}
	(b) Compute a compact minimal cover of $R$ with $\Sigma$.\vspace{10pt}
	
	\textbf{Solution}: \\\vspace{5pt}
	Given the minimal cover of $R$ with $\Sigma$:\\\vspace{3pt}
	$\{A\} \rightarrow \{B\}$\\	
	$\{A\} \rightarrow \{C\}$\\	
	$\{B\} \rightarrow \{A\}$\\
	$\{C\} \rightarrow \{D\}$\\\vspace{5pt}
	The \textbf{compact minimal cover} is:\\\vspace{3pt}
	$\{A\} \rightarrow \{B, C\}$\\	
	$\{B\} \rightarrow \{A\}$\\	
	$\{C\} \rightarrow \{D\}$
\end{frame}

\begin{frame}[fragile]{Question 2 Cont.}
Now let's look back the relation $R$ with all FDs ($\Sigma$):\\\vspace{5pt}
\begin{figure}
	\includegraphics[width=0.85\textwidth, trim=0 0 0 0, clip]{t5/images/end_q1.png}
\end{figure}

\textit{(We will keep using this (kind of) figures for later demonstrations)}
\end{frame}

\section*{Question 3 Normal Forms}

\begin{frame}[fragile]{Question 3 Normal Forms.}
(a) Is $R$ with $\Sigma$ in 2NF?\\
(b) Is $R$ with $\Sigma$ in 3NF?\\
(c) Is $R$ with $\Sigma$ in BCNF?\\\vspace{10pt}

\begin{figure}
	\includegraphics[width=0.75\textwidth, trim=0 0 0 0, clip]{t5/images/normal_forms.png}
\end{figure}
\end{frame}

\begin{frame}[fragile]{Question 3 (a)}
(a) Is $R$ with $\Sigma$ in 2NF?\\\vspace{10pt}

\textbf{Recap}: How to \textcolor{brown}{mechanically} define 2NF?\\\vspace{10pt}
A relation $R$ with a set of functional dependencies $\Sigma$ is in second norm form (2NF) if and only iff for every functional dependency $X\rightarrow\{A\}$, we have:
\begin{itemize}
	\item $X\rightarrow\{A\}$ is trivial, or
	\item $X$ is not a subset of a candidate key, or
	\item $A$ is a prime attribute.
\end{itemize}\vspace{10pt}

\begin{alertblock}{Just to understand, don't keep it in mind!}
	Unlike the 1NF that allow any attribute as long as they are single values, 
	The 2NF is more strict so that it does not allow any non-prime attribute to be determined by partial (not full) candidate keys (e.g. \texttt{\textbf{postal}} is only determined by \texttt{\textbf{email}} or \texttt{\textbf{phone}}, but not \texttt{\textbf{service\_name}}).
\end{alertblock}

\end{frame}

\begin{frame}[fragile]{Question 3 (a) Cont.}
\begin{figure}
	\includegraphics[width=0.85\textwidth, trim=0 0 0 0, clip]{t5/images/q3_2nf_highlight.png}
\end{figure}
\end{frame}

\begin{frame}[fragile]{Question 3 (a) Cont.}
\textbf{Solution}:
Let us look at the non-trivial functional dependencies of the form $X \rightarrow \{A\}$ derived from $\Sigma$. Namely after removing the trivial functional dependencies after step 1 of the minimal cover algorithm. Equivalently, we could use a minimal cover.\\\vspace{3pt}
$\{A\} \rightarrow \{C\}$ is non-trivial, $\{A\}$ is a proper subset of a candidate key and $\{C\}$ is not a prime attribute. This functional dependency violates the three conditions of the 2NF definition. $R$ with $\Sigma$ is not in 2NF.\\\vspace{3pt}

Incidentally, several other functional dependencies also violate the 2NF definition:\\
$\{B\} \rightarrow \{C\}$ is non-trivial, $\{B\}$ is a proper subset of a candidate key and $\{C\}$ is not a prime attribute. \\\vspace{3pt}
This ones do not (one condition is met):\\
$\{B,C\} \rightarrow \{D\}$ $\{B,C\}$ is not a proper subset of a candidate key.\\
$\{C\} \rightarrow \{D\}$ $\{C\}$ is not a proper subset of a candidate key.\\
$\{A\} \rightarrow \{B\}$ $\{B\}$ is a prime attribute.\\
$\{B, C\} \rightarrow \{A\}$ $\{A\}$ is a prime attribute.
\end{frame}

\begin{frame}[fragile]{Question 3 (b)}

(b) Is $R$ with $\Sigma$ in 3NF?\\\vspace{10pt}

\textbf{Recap}: How to \textcolor{brown}{mechanically} define 3NF?\\\vspace{10pt}
A relation $R$ with a set of functional dependencies $\Sigma$ is in third norm form (3NF) if and only iff for every functional dependency $X\rightarrow\{A\}$, we have:
\begin{itemize}
	\item $X\rightarrow\{A\}$ is trivial, or
	\item $X$ is a superkey, or
	\item $A$ is a prime attribute.
\end{itemize}\vspace{5pt}

\begin{alertblock}{Just to understand, don't keep it in mind!}
	Unlike the 2NF that still allows a non-prime attribute to be determined by another non-prime attribute (e.g. \texttt{\textbf{country}} is determined by \texttt{\textbf{postal}}). 
	The 3NF is more strict so that it does not allow any non-prime attribute to be determined by other non-prime attributes (all non-prime attributes have to be independent to each other, and therefore can only be determined by prime attributes).
\end{alertblock}

\end{frame}

\begin{frame}[fragile]{Question 3 (b) Cont.}
\begin{figure}
	\includegraphics[width=0.95\textwidth, trim=0 0 0 0, clip]{t5/images/q3_3nf_highlight.png}
\end{figure}
\end{frame}

\begin{frame}[fragile]{Question 3 (b) Cont.}
\textbf{Solution}:
Let us look at the non-trivial functional dependencies of the form $X \rightarrow \{A\}$ derived from $\Sigma$. Namely after removing the trivial functional dependencies after step 1 of the minimal cover algorithm. Equivalently, we could use a minimal cover.\\\vspace{3pt}
$\{A\} \rightarrow \{C\}$ is non-trivial, $\{A\}$ is not a superkey and $\{C\}$ is not a prime attribute. This functional dependency violates the three conditions of the 3NF definition. $R$ with $\Sigma$ is not in 3NF.\\\vspace{3pt}

\end{frame}

\begin{frame}[fragile]{Question 3 (b) Cont.}
Incidentally, several other functional dependencies also violate the 3NF definition:\\

$\{B, C\} \rightarrow \{D\}$ is non-trivial, $\{B, C\}$ is not a superkey and $\{D\}$ is not a prime attribute.\\
$\{B\} \rightarrow \{C\}$ is non-trivial, $\{B\}$ is not a superkey and $\{C\}$ is not a prime attribute.\\
$\{C\} \rightarrow \{D\}$ is non-trivial, $\{C\}$ is not a superkey and $\{D\}$ is not a prime attribute.\\\vspace{3pt}

This one does not (one condition is met):\\
$\{A\} \rightarrow \{B\}$ is non-trivial, $\{A\}$ is not a superkey but $\{B\}$ is a prime attribute.\\\vspace{3pt}
$\{B, C\} \rightarrow \{A\}$ $\{A\}$ is a prime attribute.\\\vspace{3pt}

\end{frame}

\begin{frame}[fragile]{Question 3 (c)}
(c) Is $R$ with $\Sigma$ in BCNF?\\\vspace{10pt}

\textbf{Recap}: How to \textcolor{brown}{mechanically} define BCNF?\\\vspace{10pt}
A relation $R$ with a set of functional dependencies $\Sigma$ is in Boyce-Codd norm form (BCNF) if and only iff for every functional dependency $X\rightarrow\{A\}$, we have:
\begin{itemize}
	\item $X\rightarrow\{A\}$ is trivial, or
	\item $X$ is a superkey.
\end{itemize}\vspace{3pt}

\begin{alertblock}{Just to understand, don't keep it in mind!}
BCNF was proposed in 1974 (3 years later than 2NF and 3NF) and was known as 3.5NF at the beginning.\\\vspace{3pt}
Unlike the 3NF that does not allow a non-prime attribute to be determined by other non-prime attributes (e.g. \texttt{country} is determined by \texttt{postal}). 
The BCNF is more strict so that it does not allow any attribute to be determined by non-prime attributes.
\end{alertblock}

\end{frame}

\begin{frame}[fragile]{Question 3 (c) Cont.}
\begin{figure}
	\includegraphics[width=0.85\textwidth, trim=0 0 0 0, clip]{t5/images/q3_bcnf_highlight.png}
\end{figure}
\end{frame}

\begin{frame}[fragile]{Question 3 (c) Cont.}
\textbf{Solution}:
$R$ with $\Sigma$ is not in 3NF, then it cannot be in BCNF.\\\vspace{3pt}

Let us look at the non-trivial functional dependencies of the form $X \rightarrow \{A\}$ derived from $\Sigma$. Namely after removing the trivial functional dependencies after step 1 of the minimal cover algorithm. Equivalently, we could use a minimal cover.\\\vspace{3pt}

$\{A\} \rightarrow \{C\}$ is non-trivial and $\{A\}$ is not a superkey . This functional dependency violates the two conditions of the BCNF definition. $R$ with $\Sigma$ is not in BCNF.\\\vspace{3pt}

Incidentally, all the other functional dependencies also violate the BCNF definition:\\
$\{A\} \rightarrow \{B\}$ is non-trivial and $\{A\}$ is not a superkey.\\
$\{B, C\} \rightarrow \{D\}$ is non-trivial and $\{B, C\}$ is not a superkey.\\
$\{B\} \rightarrow \{C\}$ is non-trivial and $\{B\}$ is not a superkey.\\
$\{C\} \rightarrow \{D\}$ is non-trivial and  $\{C\}$ is not a superkey.\\
$\{B, C\} \rightarrow \{A\}$ $\{A\}$ is non-trivial and  $\{B, C\}$ is not a superkey.
\end{frame}

\section*{Question 4 Normalisation}

\begin{frame}[fragile]{Question 4 (a)}
(a) Synthesise $R$ with $\Sigma$ into a 3NF decomposition using the algorithm from the lecture.\\\vspace{5pt}

\textbf{Solution}:
First compute a compact minimal cover:

$\{A\} \rightarrow \{B, C\}$

$\{B\} \rightarrow \{A\}$

$\{C\} \rightarrow \{D\}$

For each functional dependency create a fragment:

$R_1 = \{A, B, C\}$

$R_2 = \{B, A\}$ , this fragment is not kept: it is subsumed by $R_1$.

$R_3 = \{C, D\}$.

None of the fragments contain a candidate key. We choose one: say $\{A, E\}$, and add it as fragment:

$R_4 = \{A, E\}$.

The result is:
$R_1 = \{\underline{A}, \underline{B}, C\}$
$R_3 = \{\underline{C}, D\}$.
$R_4 = \{\underline{A, E}\}$.

The algorithm always works. It is guaranteed to find a decomposition in 3NF. Note that using a different minimal cover or compact minimal cover may give a different (but equally correct) result.
\end{frame}

\begin{frame}[fragile]{Question 4 (b-c)}
(b) Is the results lossless?\\\vspace{5pt}
\textbf{Solution}: Yes, the algorithm guarantees that the result is lossless.
\\\vspace{30pt}
(c) Is the results dependency preserving?\\\vspace{5pt}
\textbf{Solution}:
Yes, the algorithm guarantees that the result is dependency preserving.

\end{frame}

\begin{frame}[fragile]{Question 4 (d)}
(d) Is the results in BCNF?\\\vspace{5pt}

\textbf{Solution}:
The algorithm guarantees that the result is in 3NF. In this case, it is also in BCNF! (you can check). It is not always but often the case.
\end{frame}

\begin{frame}[fragile]{Question 4 (e)}
(e) Decompose $R$ with $\Sigma$ into a BCNF decomposition using the algorithm from the lecture.\\\vspace{5pt}

\textbf{Solution}: We found that $\{A\} \rightarrow \{C\}$ violates the BCNF condition.\\\vspace{3pt}

We use it to decompose into two fragments:\\
$R_1 = \{A\}^{+} = \{A, B, C, D\}$ with $\Sigma_1 = \{\{A\} \rightarrow \{C\}, \{B\} \rightarrow \{A\}, \{C\} \rightarrow \{D\}\}$.

$R_2 = \{A, E\}$ with $\Sigma_2 = \emptyset$.

$R_2$ with $\Sigma_2$ is in BCNF.

$R_1$ with $\Sigma_1$ is not in BCNF because $\{C\} \rightarrow \{D\}$ violates the BCNF condition.
\textbf{$\leftarrow$ We need to further decompose it.}
\end{frame}

\begin{frame}[fragile]{Question 4 (e) Cont.}
For $R_1 = \{A\}^{+} = \{A, B, C, D\}$, we use it to decompose into two fragments:\\\vspace{5pt}
$R_{1.1} = \{C\}^{+} = \{C, D\}$ (computed for $R_1$ with $\Sigma_1$!) with $\Sigma_{1.1} = \{\{C\} \rightarrow \{D\}\}$.

$R_{1.2} = \{A, B, C\}$ with $\Sigma_{1.2} = \{\{A\} \rightarrow \{B\}, \{A\} \rightarrow \{C\}, \{B\} \rightarrow \{A\}\}$.\\\vspace{5pt}

$R_{1.1}$ with $\Sigma_{1.1}$ is in BCNF.

$R_{1.2}$ with $\Sigma_{1.2}$ is in BCNF.\\\vspace{5pt}

The result is a decomposition into three fragments: $R_2$, $R_{1.1}$ and $R_{1.2}$. It is the same result as the synthesis.\\\vspace{5pt}

However, this (the decomposition outcome is same as the synthesis result) is not always the case. In particular, some dependencies can be lost.
\end{frame}

\begin{frame}[fragile]{Question 4 (f-g)}
(f) Is the results lossless?\\\vspace{5pt}
\textbf{Solution}: Yes, the algorithm guarantees that the result is lossless.\\
\vspace{30pt}
(g) Is the results dependency preserving?\\\vspace{5pt}
\textbf{Solution}:
Incidentally yes, the \textbf{algorithm} (itself) \textcolor{red}{\textbf{does not}} guarantee that the result is dependency preserving.\\
\begin{alertblock}{Notice}
	Don't take the dependency preserving for granted in decomposition.
\end{alertblock}

\end{frame}

\begin{frame}[fragile]{Question 4 Cont. \textcolor{red}{(Extra)}}
To wrap-up, BCNF decomposition is \textbf{not always optimal}, as it aims to dismiss some ``unnecessary'' dependencies among attributes. (sometimes 3NF is good enough)\\\vspace{5pt}
Let's look back the decomposition result:

$R_{1.1}=\{C, D\}$ (\textbf{\underline{postal}}, \textbf{country})

$R_{1.1}=\{A, B, C\}$ (\textbf{\underline{mobile}}, \textbf{\underline{email}}, \textbf{postal})

$R_{2}=\{A, E\}$ (\underline{\textbf{mobile}, \textbf{service}})\\\vspace{5pt}

\textbf{Extra}:
How to compare with ER design (the previous tutorial)? ER design and Normalisation (historically) uses to be competitive relationship in database design \& tuning, with the exactly same goal.\\\vspace{5pt}
\end{frame}

\begin{frame}[fragile]{Question 4 Cont. \textcolor{red}{(Extra)}}

C.f. the decomposition outcome: \\
(\textbf{\underline{postal}}, \textbf{country}),
(\textbf{\underline{mobile}}, \textbf{\underline{email}}, \textbf{postal}),
and (\underline{\textbf{mobile}, \textbf{service}})\\\vspace{5pt}

Take a look at the ER diagram and DDL derived based on Q1 story:\\\vspace{5pt}
\begin{figure}
	\includegraphics[width=0.6\textwidth, trim=0 0 0 0, clip]{t5/images/er-diagram.png}
\end{figure}

Can you spot any problem in the design above?\\\vspace{5pt}
\end{frame}

\begin{frame}[fragile]{Question 4 Cont. \textcolor{red}{(Extra)}}
We find that ER design is a \textbf{top-down} approach, while normalisation is the \textbf{bottom-up} one.\\\vspace{5pt}

\begin{block}{ER design vs. Normalisation}
	In ER design, we always group first (identify entity and relationship sets), then go into details (identify the tables and their attributes).\\
	While in normalisation, we always focus on details first (add all attributes in a flatten form with functional dependencies), then group them into proper relations (based on the specific normal form stipulated).
\end{block}

A wise \& modern strategy is to take advantage of both, although it might be hard (not difficult for human, but hard for computer), i.e.,
perform ER design first, and make it compliant with a desired norm form.
\end{frame}

\section*{Extra Practice}

\begin{frame}[fragile]{\boss{Extra Practice}}
	Challenging questions ahead!!! I share with you 4 practice questions and each one may cost you 5-10 mins.\\\vspace{10pt}
	\textcolor{brown}{If you can UNDERSTAND their answers, you basically uderstand all key knowledge of this tutorial and you should be fine for the exam/test.}\\\vspace{10pt}
	\textcolor{brown}{If you can easily SOLVE these questions, you are a master of normalization (at least for the scope of this module) and you can teach this part in the future.}\\
	
	\begin{columns}[t]
	\column{0.65\textwidth}
	\\\vspace{5pt}
	\textbf{Free lunch, why not take it?}\\
	\textbf{Come on! Die-die must try!}\\\vspace{10pt}
	(Due to time limitation, solutions will be posted on Luminus and elaborated during zoom T5 session (Wed 8:30-9:30 PM) so that you can watch the video recording)
	\column{0.25\textwidth}
	\begin{figure}
		\includegraphics[width=1\textwidth, trim=0 0 0 0, clip]{t5/images/mario.png}
	\end{figure}
\end{columns}
\end{frame}

% \begin{comment}
\begin{frame}[fragile]{\boss{Extra Exercise}}
	There are four extra exercises for you to practice what have learned in normalisation lecture. 
	As normalisation is one of the most challenging parts of the fundamental database module, it would be helpful for you to evaluate your understanding of this part. \\
\end{frame}
\end{comment}

\begin{frame}[fragile]{\boss{Extra} - Case 1}
	We are designing a warehouse management system for a lot of warehouses. Each warehouse ($W$) has one manager ($M$), and each manager only manage one warehouse. There could be many products ($P$) in per warehouse. For each product we also record its stock number ($S$).\\\vspace{10pt}
	\textbf{Questions}:\\
	(1) Find \textbf{candidate key(s)} and \textbf{prime attribute(s)} from attribute closures $\Sigma^{+}$.\\
	(2) Compute the \textbf{compact minimal cover} of $R$ with all FDs $\Sigma$.\\
	(3) Determine if it is \textbf{2NF}? If yes, is it \textbf{3NF}? If yes, is it \textbf{BCNF}?\\
	(4) If it is not 3NF, \textbf{synthesis} the relations to make it 3NF.\\
	(5) If it is not BCNF, \textbf{decomposite} the relations to make it BCNF and verify the \textbf{dependency preservation}. 
\end{frame}

\begin{frame}[fragile]{\boss{Extra} - Case 2}
	We are designing a transcript issuing system for our university. 
	Each student is identified by its matric number/student ID, written as $S$.
	We are going to record a grade ($G$) for each student ($S$) and each module ($M$).
	We also record students' names ($N$) and their faculty ($F$). In case any verification is needed, we also save the dean's name for each department ($D$) so that people can contact him/her.\\\vspace{10pt}
	\textbf{Questions}:\\
	(1) Find \textbf{candidate key(s)} and \textbf{prime attribute(s)} from attribute closures $\Sigma^{+}$.\\
	(2) Compute the \textbf{compact minimal cover} of $R$ with all FDs $\Sigma$.\\
	(3) Determine if it is \textbf{2NF}? If yes, is it \textbf{3NF}? If yes, is it \textbf{BCNF}?\\
	(4) If it is not 3NF, \textbf{synthesis} the relations to make it 3NF.\\
	(5) If it is not BCNF, \textbf{decomposite} the relations to make it BCNF and verify the \textbf{dependency preservation}. 
\end{frame}

\begin{frame}[fragile]{\boss{Extra} - Case 3}
	This time we deal with abstract relations with functional dependencies shown as in the figure below:\\\vspace{-5pt}
	
	\begin{figure}
		\includegraphics[width=0.4\textwidth, trim=0 0 0 0, clip]{t5/images/case3.png}
	\end{figure}\vspace{-10pt}
	
	\textbf{Questions}:\\
	(1) Find \textbf{candidate key(s)} and \textbf{prime attribute(s)} from attribute closures $\Sigma^{+}$.\\
	(2) Compute the \textbf{compact minimal cover} of $R$ with all FDs $\Sigma$.\\
	(3) Determine if it is \textbf{2NF}? If yes, is it \textbf{3NF}? If yes, is it \textbf{BCNF}?\\
	(4) If it is not 3NF, \textbf{synthesis} the relations to make it 3NF.\\
	(5) If it is not BCNF, \textbf{decomposite} the relations to make it BCNF and verify the \textbf{dependency preservation}. 
\end{frame}

\begin{frame}[fragile]{\boss{Extra} - Case 4}
	Another abstract relations with functional dependencies shown as in the figure below:\\\vspace{-5pt}
	
	\begin{figure}
		\includegraphics[width=0.25\textwidth, trim=0 0 0 0, clip]{t5/images/case4.png}
	\end{figure}\vspace{-5pt}
	
	\textbf{Questions}:\\
	(1) Find \textbf{candidate key(s)} and \textbf{prime attribute(s)} from attribute closures $\Sigma^{+}$.\\
	(2) Compute the \textbf{compact minimal cover} of $R$ with all FDs $\Sigma$.\\
	(3) Determine if it is \textbf{2NF}? If yes, is it \textbf{3NF}? If yes, is it \textbf{BCNF}?\\
	(4) If it is not 3NF, \textbf{synthesis} the relations to make it 3NF.\\
	(5) If it is not BCNF, \textbf{decomposite} the relations to make it BCNF and verify the \textbf{dependency preservation}. 
\end{frame}



\section*{Farewell}

\begin{frame}{}
	\begin{figure}
		\includegraphics[width=0.6\textwidth, trim=0 0 0 0, clip]{t5/images/final.png}
	\end{figure}
\end{frame}
\begin{frame}{}
	\centering  
	For any further question, please feel free to email me:\vspace{10pt}
	
	huasong.meng@u.nus.edu\\\vspace{3pt}
	menghs@i2r.a-star.edu.sg (work)\\\vspace{3pt}
	huasong.meng@gmail.com (personal)\vspace{10pt}
	
	Or you can whatsapp/wechat me via: 81028639 \vspace{20pt}
	
	\begin{tcolorbox}
		\begin{center}
			\textcolor{red}{Copyright 2021 Mark H. Meng. All rights reserved.}
		\end{center}
	\end{tcolorbox}
\end{frame}