
\title{CS4221/CS5421}

\subtitle{Normal Forms and Normalisation\\ \textbf{(Extra Practice)}}

\author{Mark Meng Huasong}

\institute[National University of Singapore] % (optional, but mostly needed)
{
	School of Computing\\
	National University of Singapore
}

\titlegraphic{
	\includegraphics[width=2cm]{nus-logo}
}

\date{Week 7, 2022 Spring}

\begin{frame}
	\titlepage
	\begin{tcolorbox}
		\begin{center}
			{\scriptsize \textcolor{red}{All the materials within presentation slides are protected by copyrights.\\
					It is forbidden by NUS to upload these materials to the Internet.}}
		\end{center}
	\end{tcolorbox}
\end{frame}

\begin{frame}
	{\large \textbf{Agenda}}
	\begin{itemize}
		\item Extra Case No.1 - Warehouse management system
		\item Extra Case No.2 - University transcript issuing system
		\item Extra Case No.3 (Abstract) - \textcolor{red}{\textit{\footnotesize Non dependency preserving decomposition}}
		\item Extra Case No.4 (Abstract) - \textcolor{red}{\textit{\footnotesize Candidate keys with different sizes}}
		\item Extra Case No.5 (Abstract) 
		\item Extra Case No.6 (Abstract) - \textcolor{red}{\textit{\footnotesize Trick in finding candidate key(s)}}
		
	\end{itemize}\vspace{20pt}

	%\textcolor{red}{\textbf{\hl{$\ddagger$ Updated in the latest version as some errors raised in-class have been fixed}}}
\end{frame}

\begin{frame}[fragile]{\boss{Extra} - Case 1}
	We are designing a warehouse management system for a lot of warehouses. Each warehouse ($W$) has one manager ($M$), and each manager only manage one warehouse. There could be many products ($P$) in per warehouse. For each product we also record its stock number ($S$).\\\vspace{10pt}
	\textbf{Questions}:\\
	(1) Find \textbf{candidate key(s)} and \textbf{prime attribute(s)} from attribute closures $\Sigma^{+}$.\\
	(2) Compute the \textbf{compact minimal cover} of $R$ with all FDs $\Sigma$.\\
	(3) Determine if it is \textbf{2NF}? If yes, is it \textbf{3NF}? If yes, is it \textbf{BCNF}?\\
	(4) If it is not 3NF, \textbf{synthesis} the relations to make it 3NF.\\
	(5) If it is not BCNF, \textbf{decomposite} the relations to make it BCNF and verify the \textbf{dependency preservation}. 
\end{frame}

\begin{frame}[fragile]{\boss{Extra} - Case 1 (Solution)}
	\textbf{W}: warehouse; \textbf{M}: manager;
	\textbf{P}: product; \textbf{S}: stock.\\\vspace{5pt}
	$R = \{W, M, P, S\}$\\
	$\Sigma = \{\{W\} \rightarrow \{M\}, \{M\} \rightarrow \{W\},
	\{W, P\} \rightarrow \{S\}\}$.\\\vspace{15pt}
	\begin{figure}
		\includegraphics[width=0.2\textwidth, trim=0 0 0 0, clip]{t5/images/case1.png}
	\end{figure}\vspace{-5pt}
\end{frame}

\begin{frame}[fragile]{\boss{Extra} - Case 1 (Solution)}
	$R = \{W, M, P, S\}$\\
	$\Sigma = \{\{W\} \rightarrow \{M\}, \{M\} \rightarrow \{W\},
	\{W, P\} \rightarrow \{S\}\}$.\\\vspace{5pt}
	
	\textbf{Solution}:\\
	(1) The attribute closure of $R$ is:
	\begin{align*} 
		\Sigma^{+} = \{&\{W\}^{+} \rightarrow \{W, M\},\\
		&\{M\}^{+} \rightarrow \{W, M\},\\
		&\{P\}^{+} \rightarrow \{P\},\\
		&\{S\}^{+} \rightarrow \{S\},\\
		&\{{M,W}\}^{+} \rightarrow \{W, M\},\\
		&\{{M,P}\}^{+} \rightarrow \{W, M, P, S\},\\
		&\{{M,S}\}^{+} \rightarrow \{W, M, S\},\\
		&\{{W,P}\}^{+} \rightarrow \{W, M, P, S\},\\
		&\{{W,S}\}^{+} \rightarrow \{W, M, S\},...\}.
	\end{align*} 
	
	Now we find candidate keys: $\{W, P\}$ and $\{M, P\}$.\\
	Prime attributes: $W, M, P$
\end{frame}

\begin{frame}[fragile]{\boss{Extra} - Case 1 (Solution)}
	(2) The compact minimal cover is:\\\vspace{5pt}
	
	$\{W\} \rightarrow \{M\},$\\
	$\{M\}  \rightarrow \{W\},$\\
	$\{W, P\} \rightarrow \{S\}.$\\\vspace{5pt}
	
	(3)  Yes it is 2NF, 3NF, but not BCNF (e.g., $\{W\} \rightarrow \{M\}$, where $\{W\}$ is not a superkey).\\\vspace{5pt}
	
	(4) Omitted as it is 3NF. \\\vspace{5pt}
	
	(5) Decomposition at $\{W\} \rightarrow \{M\}$:\\\vspace{5pt}
	
	$R_1 = (\underline{W}, \underline{M})$, with $\Sigma_1 =\{\{W\} \rightarrow \{M\}, \{M\}  \rightarrow \{W\}\}$\\
	$R_2 = (\underline{W,P}, S)$, with $\Sigma_2 =\{\{W,P\} \rightarrow \{S\}\}$\\ \vspace{5pt}
	
	It is (luckily) dependency preserving.
	
\end{frame}

\begin{frame}[fragile]{\boss{Extra} - Case 2}
	We are designing a transcript issuing system for our university. 
	Each student is identified by its matric number/student ID, written as $S$.
	We are going to record a grade ($G$) for each student ($S$) and each module ($M$).
	We also record students' names ($N$) and their faculty ($F$). In case any verification is needed, we also save the dean's name for each department ($D$) so that people can contact him/her.\\\vspace{10pt}
	\textbf{Questions}:\\
	(1) Find \textbf{candidate key(s)} and \textbf{prime attribute(s)} from attribute closures $\Sigma^{+}$.\\
	(2) Compute the \textbf{compact minimal cover} of $R$ with all FDs $\Sigma$.\\
	(3) Determine if it is \textbf{2NF}? If yes, is it \textbf{3NF}? If yes, is it \textbf{BCNF}?\\
	(4) If it is not 3NF, \textbf{synthesis} the relations to make it 3NF.\\
	(5) If it is not BCNF, \textbf{decomposite} the relations to make it BCNF and verify the \textbf{dependency preservation}. 
\end{frame}

\begin{frame}[fragile]{\boss{Extra} - Case 2 (Solution)}
	\textbf{S}: student ID; \textbf{M}: module; \textbf{N}: name;
	\textbf{F}: faculty; \textbf{G}: grade; \textbf{D}: dean\\\vspace{5pt}
	$R = \{S, M, G, N, F, D\}$\\
	$\Sigma = \{\{S, M\} \rightarrow \{G\}, 
	\{S\} \rightarrow \{N, F\},
	\{F\} \rightarrow \{D\}\}$.\\\vspace{15pt}
	\begin{figure}
		\includegraphics[width=0.4\textwidth, trim=0 0 0 0, clip]{t5/images/case2.png}
	\end{figure}\vspace{-5pt}
\end{frame}

\begin{frame}[fragile]{\boss{Extra} - Case 2 (Solution)}
	$R = \{S, M, G, N, F, D\}$\\
	$\Sigma = \{\{S, M\} \rightarrow \{G\}, 
	\{S\} \rightarrow \{N, F\},
	\{F\} \rightarrow \{D\}\}$.\\\vspace{5pt}
	
	\textbf{Solution}:\\
	(1) The attribute closure of $R$ is:
	\begin{align*} 
		\Sigma^{+} = \{&\{S\}^{+} \rightarrow \{S, N, F, D\},\\
		&\{M\}^{+} \rightarrow \{M\},\\
		&\{G\}^{+} \rightarrow \{G\},\\
		&\{N\}^{+} \rightarrow \{N\},\\
		&\{F\}^{+} \rightarrow \{F, D\},\\
		&\{D\}^{+} \rightarrow \{D\},\\
		&\{{S, M}\}^{+} \rightarrow \{S, M, N, F, D, G\},\\
		&\{{S, G}\}^{+} \rightarrow \{S, G, N, F, D\}, ...
	\end{align*} 
	
	Now we find candidate keys: $\{S, M\}$.\\
	Prime attributes: $S, M$
\end{frame}

\begin{frame}[fragile]{\boss{Extra} - Case 2 (Solution)}
	(2) The compact minimal cover is:\\\vspace{5pt}
	
	$\{S, M\} \rightarrow \{G\},$\\
	$\{S\}  \rightarrow \{N, F\},$\\
	$\{F\} \rightarrow \{D\}.$\\\vspace{5pt}
	
	(3)  No it is not 2NF (e.g., $\{S\}\rightarrow\{N, F\}$, \{N, F\} are \textbf{not} prime attributes and $S$ \textbf{is} a subset of candidate key). Therefore it is not 3NF, and not BCNF, too.\\
	(Besides $\{S\}\rightarrow\{N, F\}$, $\{F\}\rightarrow\{D\}$ violates 3NF and BCNF, too)
	\vspace{5pt}
	
	(4) Synthesis result has 3 relations and is (luckily) BCNF: \\\vspace{5pt}
	
	$R_1 = (\underline{S,M}, G)$, with $\Sigma_1 =\{\{S,M\} \rightarrow \{G\}\}$\\
	$R_2 = (\underline{S}, N, F)$, with $\Sigma_2 =\{\{S\} \rightarrow \{N,F\}\}$\\
	$R_3 = (\underline{F}, D)$, with $\Sigma_3 =\{\{F\} \rightarrow \{D\}\}$\\\vspace{5pt}
	
\end{frame}

\begin{frame}[fragile]{\boss{Extra} - Case 2 (Solution)}
	(5) Decomposition at $\{F\} \rightarrow \{D\}$:\\\vspace{5pt}
	
	$R_1 = (\underline{F}, D)$, with $\Sigma_1 =\{\{F\} \rightarrow \{D\}\}$, which is in BCNF\\
	$R_2 = (\underline{S,M}, G, N, F)$, with $\Sigma_2 =\{\{S,M\} \rightarrow \{G\}, \{S\} \rightarrow \{N,F\}\}$\\\vspace{5pt}
	
	The $\{S\} \rightarrow \{N,F\}$ violates BCNF because the LHS is not a superkey, therefore $R_2$ needs to be further decomposed, at $\{S\} \rightarrow \{N,F\}$:\\\vspace{5pt}
	
	$R_{2.1} = (\underline{S}, N, F)$, with $\Sigma_{2.1} =\{\{S\} \rightarrow \{N,F\}\}$\\
	$R_{2.2} = (\underline{S,M}, G)$, with $\Sigma_{2.2} =\{\{S,M\} \rightarrow \{G\}\}$\\\vspace{5pt}
	
	As the result, the BCNF decomposition is (luckily) dependency preserving and is given below:\\\vspace{5pt}
	
	$R_1 = (\underline{F}, D),$\\
	$R_{2.1} = (\underline{S}, N, F).$\\
	$R_{2.2} = (\underline{S,M}, G).$\\\vspace{5pt}
	
	\textit{(In fact this decomposition result is same with the synthesis one)}
\end{frame}

\begin{frame}[fragile]{\boss{Extra} - Case 3}
	This time we deal with abstract relations with functional dependencies shown as in the figure below:\\\vspace{-5pt}
	
	\begin{figure}
		\includegraphics[width=0.4\textwidth, trim=0 0 0 0, clip]{t5/images/case3.png}
	\end{figure}\vspace{-10pt}
	
	\textbf{Questions}:\\
	(1) Find \textbf{candidate key(s)} and \textbf{prime attribute(s)} from attribute closures $\Sigma^{+}$.\\
	(2) Compute the \textbf{compact minimal cover} of $R$ with all FDs $\Sigma$.\\
	(3) Determine if it is \textbf{2NF}? If yes, is it \textbf{3NF}? If yes, is it \textbf{BCNF}?\\
	(4) If it is not 3NF, \textbf{synthesis} the relations to make it 3NF.\\
	(5) If it is not BCNF, \textbf{decomposite} the relations to make it BCNF and verify the \textbf{dependency preservation}. 
\end{frame}


\begin{frame}[fragile]{\boss{Extra} - Case 3 (Solution)}
	
	\textbf{Solution}:\\
	(1) The attribute closure of $R = \{A, B, C, D\}$ is:\\\vspace{5pt}
	\begin{scriptsize}
		\begin{align*} 
			\Sigma^{+} = \{&\{A\}^{+} \rightarrow \{A\},\\
			&\{B\}^{+} \rightarrow \{B\},\\
			&\{C\}^{+} \rightarrow \{A, C, D\},\\
			&\{D\}^{+} \rightarrow \{A, D\},\\
			&\{A, B\}^{+} \rightarrow \{A, B, C, D\},\\
			&\{A, C\}^{+} \rightarrow \{A, C, D\},\\
			&\{A, D\}^{+} \rightarrow \{A, D\},\\
			&\{B, C\}^{+} \rightarrow \{A, B, C, D\}\\
			&\{B, D\}^{+} \rightarrow \{A, B, C, D\}, ...
		\end{align*}
	\end{scriptsize}
	
	Now we find candidate keys: $\{A, B\}$ or $\{B, C\}$ or $\{B, D\}$.\\
	Prime attributes: $A, B, C, D$ (There is no non-prime attribute for this case)
\end{frame}

\begin{frame}[fragile]{\boss{Extra} - Case 3 (Solution)}
	(2) The compact minimal cover is:\\\vspace{5pt}
	
	$\{A, B\} \rightarrow \{C\}$\\
	$\{C\} \rightarrow \{D\}$\\
	$\{D\} \rightarrow \{A\}$.\\\vspace{5pt}
	
	(3) Yes it is 2NF and 3NF (because all attributes are prime attributes). However, it is not BCNF (e.g., $\{C\} \rightarrow \{D\}$ and $\{D\} \rightarrow \{A\}$).\\\vspace{5pt}
	
	(4) Omitted as it is 3NF. \\\vspace{5pt}
	
\end{frame}

\begin{frame}[fragile]{\boss{Extra} - Case 3 (Solution)}
	(5) Decomposition at $\{C\} \rightarrow \{D\}$:\\\vspace{3pt}
	
	$R_1 = (A, \underline{C}, D)$, with $\Sigma_1 =\{\{C\} \rightarrow \{D\}, \{D\} \rightarrow \{A\}\}$\\
	$R_2 = (\underline{B,C})$, with $\Sigma_2 =\emptyset$\\\vspace{3pt}
	
	The $R_1$ is not in BCNF (e.g., $\{D\} \rightarrow \{A\}$, the LHS is not superkey), let's further decompose it:\\
	$R_{1.1} = (A, \underline{D})$, with $\Sigma_{1.1} =\{\{D\} \rightarrow \{A\}\}$\\
	$R_{1.2} = (\underline{C}, D)$, with $\Sigma_{1.2} =\{\{C\} \rightarrow \{D\}\}$\\\vspace{3pt}
	
	In this way all 3 relations ($R_{1.1}$, $R_{1.2}$ and $R_{2}$) are BCNF, but we lose a dependency $\{A, B\} \rightarrow \{C\}$.\\\vspace{10pt}
	
	\textcolor{red}{How about we change the entry point of decomposition?}\\\vspace{3pt}
	
	Decomposition at $\{D\} \rightarrow \{A\}$:\\\vspace{3pt}
	
	$R_1 = (A, \underline{D}),$\\
	$R_2 = (\underline{B,C}, D).$\\\vspace{5pt}
	Then $R_2$ needs to be further decomposed to $R_{2}.1 = (\underline{C}, D), R_{2.2} = (\underline{B,C}).$ to ensure all of them are BCNF. However, we (still) lose that dependency $\{A, B\} \rightarrow \{C\}$.
\end{frame}

\begin{frame}[fragile]{\boss{Extra} - Case 4}
	Another abstract relations with functional dependencies shown as in the figure below:\\\vspace{-5pt}
	
	\begin{figure}
		\includegraphics[width=0.25\textwidth, trim=0 0 0 0, clip]{t5/images/case4.png}
	\end{figure}\vspace{-5pt}
	
	\textbf{Questions}:\\
	(1) Find \textbf{candidate key(s)} and \textbf{prime attribute(s)} from attribute closures $\Sigma^{+}$.\\
	(2) Compute the \textbf{compact minimal cover} of $R$ with all FDs $\Sigma$.\\
	(3) Determine if it is \textbf{2NF}? If yes, is it \textbf{3NF}? If yes, is it \textbf{BCNF}?\\
	(4) If it is not 3NF, \textbf{synthesis} the relations to make it 3NF.\\
	(5) If it is not BCNF, \textbf{decomposite} the relations to make it BCNF and verify the \textbf{dependency preservation}. 
\end{frame}

\begin{frame}[fragile]{\boss{Extra} - Case 4 (Solution)}
	
	\textbf{Solution}:\\
	(1) The attribute closure of $R = \{A, B, C, D\}$ is:\\
	\begin{scriptsize}
		\begin{align*} 
			\Sigma^{+} = \{&\{A\}^{+} \rightarrow \{A, B, C, D\},\\
			&\{B\}^{+} \rightarrow \{B, D\},\\
			&\{C\}^{+} \rightarrow \{C\},\\
			&\{D\}^{+} \rightarrow \{D\},\\
			&\{A, B\}^{+} \rightarrow \{A, B, C, D\},\\
			&\{A, C\}^{+} \rightarrow \{A, B, C, D\} \text{(not minimal)},\\
			&\{A, D\}^{+} \rightarrow \{A, B, C, D\} \text{(not minimal)},\\
			&\{B, C\}^{+} \rightarrow \{A, B, C, D\}\\
			&\{B, D\}^{+} \rightarrow \{B, D\}, ...
		\end{align*}
	\end{scriptsize}\vspace{-15pt}
	
	Now we find candidate keys: $\{A\}$ or $\{B, C\}$.\\
	\textcolor{brown}{Although $\{A\}\rightarrow\{B, C\}$, both of them are candidate keys because (1) their closures functionally determine all attributes of $R$; (2) both of them are minimal superkeys (cannot be further simplified).}\\
	Prime attributes: $A, B, C$.
\end{frame}

\begin{frame}[fragile]{\boss{Extra} - Case 4 (Solution)}
	(2) First let's list all FDs given in the question:\\\vspace{5pt}
	$\{A\} \rightarrow \{B,C,D\}$, 
	$\{B\} \rightarrow \{D\}$, 
	$\{B, C\} \rightarrow \{A\}$.\\\vspace{5pt}
	
	Step 1: Simplify the RHS:\\\vspace{5pt}
	$\{A\} \rightarrow \{B\}$\\
	$\{A\} \rightarrow \{C\}$\\
	$\{A\} \rightarrow \{D\}$\\
	$\{B\} \rightarrow \{D\}$\\
	$\{B, C\} \rightarrow \{A\}$.\\\vspace{5pt}
	
	Step 2: Simplify the LHS \\
	(There is only 1 FD with multiple attributes on the LHS and that FD cannot be simplified as there does not exist any FD implies $A$) \\\vspace{5pt}
	
\end{frame}

\begin{frame}[fragile]{\boss{Extra} - Case 4 (Solution)}
	Step 3: Simplify the set:\\\vspace{5pt}
	
	$\{A\} \rightarrow \{B\}$\\
	$\{A\} \rightarrow \{C\}$\\
	$\cancel{\{A\} \rightarrow \{D\}}$ (because $\{A\} \rightarrow \{B\}$ and $\{B\} \rightarrow \{D\}$)\\
	$\{B\} \rightarrow \{D\}$\\
	$\{B, C\} \rightarrow \{A\}$.\\\vspace{5pt}
	
	Finally, the compact minimal cover is:\\\vspace{5pt}
	
	$\{A\} \rightarrow \{B, C\}$\\
	$\{B\} \rightarrow \{D\}$\\
	$\{B, C\} \rightarrow \{A\}$.\\\vspace{5pt}
	
\end{frame}

\begin{frame}[fragile]{\boss{Extra} - Case 4 (Solution)}
	(3) No it is not 2NF (e.g., $\{B\} \rightarrow \{D\}$, where $D$ is a non-prime attribute but $B$ is a subset of candidate key). Therefore, it is not 3NF and not BCNF, too. (all because of $\{B\} \rightarrow \{D\}$)\\\vspace{5pt}
	
	(4) Synthesis result has 2 relations and each one is (luckily) BCNF: \\\vspace{5pt}
	
	$R_1 = (\underline{A}, \underline{B, C}),$\\\vspace{2pt}
	$R_2 = (\underline{B}, D),$\\\vspace{1pt}
	$\cancel{R_3 = (\underline{B, C}, \underline{A})}.$ (duplicate with $R_1$)\\\vspace{5pt}
	
\end{frame}

\begin{frame}[fragile]{\boss{Extra} - Case 4 (Solution)}
	(5) Decomposition at $\{B\} \rightarrow \{D\}$:\\\vspace{5pt}
	
	$R_1 = (\underline{B}, D),$\\
	$R_2 = (\underline{A}, \underline{B, C}).$\\\vspace{5pt}
	
	In this way both relations are BCNF, and luckily we obtain the exact same result as the 3NF synthesis, therefore it is dependency preserving.\\\vspace{5pt}
	
\end{frame}

\begin{frame}[fragile]{\boss{Extra} - Case 5}
	Given a relation $R=\{A, B, C\}$ with functional dependency set:\\ 
	$\Sigma=\{\{A\} \rightarrow \{B\},\{B\} \rightarrow \{C\}, \{A, B\} \rightarrow \{C\},\{B, C\} \rightarrow \{A\}\}.$\\\vspace{10pt}
	\textbf{Questions}:\\
	(1) Find \textbf{candidate key(s)} and \textbf{prime attribute(s)} from attribute closures $\Sigma^{+}$.\\
	(2) Compute the \textbf{compact minimal cover} of $R$ with all FDs $\Sigma$.\\
	(3) Determine if it is \textbf{2NF}? If yes, is it \textbf{3NF}? If yes, is it \textbf{BCNF}?\\
	(4) If it is not 3NF, \textbf{synthesis} the relations to make it 3NF.\\
	(5) If it is not BCNF, \textbf{decomposite} the relations to make it BCNF and verify the \textbf{dependency preservation}. 
\end{frame}

\begin{frame}[fragile]{\boss{Extra} - Case 5 (Solution)}
	$R=\{A, B, C\}$\\ 
	$\Sigma=\{\{A\} \rightarrow \{B\},\{B\} \rightarrow \{C\}, \{A, B\} \rightarrow \{C\},\{B, C\} \rightarrow \{A\}\}.$\\\vspace{5pt}
	
	\textbf{Solution}:\\
	(1) The attribute closure of $R$ is:
	\begin{align*} 
		\Sigma^{+} = \{&\{A\}^{+} \rightarrow \{A,B,C\},\\
		&\{B\}^{+} \rightarrow \{A,B,C\},\\
		&\{C\}^{+} \rightarrow \{C\},\\
		&\{A,B\}^{+} \rightarrow \{A,B,C\},\\
		&\{A,C\}^{+} \rightarrow \{A,B,C\},\\
		&\{B,C\}^{+} \rightarrow \{A,B,C\},\\
		&\{A,B,C\}^{+} \rightarrow \{A,B,C\}\}.
	\end{align*} 
	
	Now we find candidate keys: $\{A\}$ and $\{B\}$.\\
	Prime attributes: $A, B$
\end{frame}

\begin{frame}[fragile]{\boss{Extra} - Case 5 (Solution)}
	$\Sigma=\{\{A\} \rightarrow \{B\},\{B\} \rightarrow \{C\}, \{A, B\} \rightarrow \{C\},\{B, C\} \rightarrow \{A\}\}.$\\\vspace{25pt}
	
	(2) The minimal cover is:\\\vspace{5pt}
	
	$\{A\} \rightarrow \{B\},$\\
	$\{B\}  \rightarrow \{C\},$\\
	$\{B\} \rightarrow \{A\}.$\\\vspace{5pt}
	
	The compact minimal cover is $\{A\} \rightarrow \{B\}, \{B\}  \rightarrow \{A,C\}$.\\\vspace{5pt}
	
	(3)  Yes it is 2NF, 3NF, and BCNF, too.\\\vspace{2pt}
	\textcolor{red}{(Hint: All LHS are superkeys)}\\\vspace{5pt}
	
	(4) Omitted as it is 3NF. \\\vspace{5pt}
	
	(5) Omitted as it is BCNF. \\\vspace{5pt}
	
\end{frame}

\begin{frame}[fragile]{\boss{Extra} - Case 6}
	Given a relation $R=\{A, B, C, D, E\}$ with functional dependency set:\\ 
	$\Sigma=\{\{C,D\} \rightarrow \{E\},\{A,B\} \rightarrow \{B\}, \{A,C,D\} \rightarrow \{E\},\{A\} \rightarrow \{E\},\{D,E\} \rightarrow \{B,C\},\{A\} \rightarrow \{A\}\}.$\\\vspace{10pt}
	\textbf{Questions}:\\
	(1) Find \textbf{candidate key(s)} and \textbf{prime attribute(s)} from attribute closures $\Sigma^{+}$.\\
	(2) Compute the \textbf{compact minimal cover} of $R$ with all FDs $\Sigma$.\\
	(3) Determine if it is \textbf{2NF}? If yes, is it \textbf{3NF}? If yes, is it \textbf{BCNF}?\\
	(4) If it is not 3NF, \textbf{synthesis} the relations to make it 3NF.\\
	(5) If it is not BCNF, \textbf{decomposite} the relations to make it BCNF and verify the \textbf{dependency preservation}. 
	\\
	\textcolor{red}{(This question's BCNF decomposition requests some tricks. It is out of the scope so don't worry if you feel hard to understand)}
\end{frame}

\begin{frame}[fragile]{\boss{Extra} - Case 6 (Solution)}
	$\Sigma=\{\{C,D\} \rightarrow \{E\},\{A,B\} \rightarrow \{B\}, \{A,C,D\} \rightarrow \{E\},\{A\} \rightarrow \{E\},\{D,E\} \rightarrow \{B,C\},\{A\} \rightarrow \{A\}\}.$\\\vspace{5pt}
	
	\textbf{Solution}:\\
	(1) The attribute closure of $R$ is:
	\vspace{-10pt}\begin{columns}
	\column{0.5\textwidth}
	\begin{scriptsize}\begin{align*} 
		\Sigma^{+} = \{&\{A\}^{+} \rightarrow \{A,E\},\\
		&\{B\}^{+} \rightarrow \{B\},\\
		&\{C\}^{+} \rightarrow \{C\},\\
		&\{D\}^{+} \rightarrow \{D\},\\
		&\{E\}^{+} \rightarrow \{E\},\\
		&\{A,B\}^{+} \rightarrow \{A,B,E\},\\
		&\{A,C\}^{+} \rightarrow \{A,C,E\},\\
		&\{A,D\}^{+} \rightarrow \{A,B,C,D,E\},\\
		&\{A,E\}^{+} \rightarrow \{A,E\}\}\\
		& ...\\
		&\{A,C,D,E\}^{+} \rightarrow \{A,B,C,D,E\}\}\\
		&\{A,B,C,D,E\}^{+} \rightarrow \{A,B,C,D,E\}\}.
	\end{align*}\end{scriptsize} 
	
	\column{0.45\textwidth}
	\textbf{Hint}: 
	After removing all trivial FDs, we find $A$ and $D$ have never appeared in the right-hand side, which means the candidate key must contains $A$ and $D$. We just need to find the minimal set that implies $\{B,C,E\}$. (Luckily both $A$ and $D$ together can determine $\{B,C,E\}$)\\\vspace{3pt}

	Then we can find candidate key: $\{A,D\}$. And the prime attributes: $A, D$
	\end{columns}
	
\end{frame}

\begin{frame}[fragile]{\boss{Extra} - Case 6 (Solution)}
	$\Sigma=\{\{C,D\} \rightarrow \{E\},\{A,B\} \rightarrow \{B\}, \{A,C,D\} \rightarrow \{E\},\{A\} \rightarrow \{E\},\{D,E\} \rightarrow \{B,C\},\{A\} \rightarrow \{A\}\}.$\\\vspace{5pt}
	
	(2) The compact minimal cover is:\\\vspace{5pt}
	
	\begin{columns}[t]
	\column{0.29\textwidth} \textbf{STEP 1}\\
	\begin{block}{}
		\tiny \textcolor{red}{Simplify the RHS by the Union Rule, i.e., if $\{X\}\rightarrow \{YZ\}$, then \underline{split} into $\{X\}\rightarrow \{Y\}$ and $\{X\}\rightarrow \{Z\}$}	
	\end{block}
	$\{C,D\} \rightarrow \{E\},$\\
	$\{A,B\} \rightarrow \{B\}$,\\
	$\{A\}  \rightarrow \{E\},$\\
	$\{A,C,D\} \rightarrow \{E\},$\\
	$\{D,E\} \rightarrow \{B\}.$\\
	$\{D,E\} \rightarrow \{C\}.$\\
	$\{A\} \rightarrow \{A\}$\\\vspace{5pt}
	\column{0.29\textwidth} \textbf{STEP 2}\\
	\begin{block}{}
		\tiny \textcolor{red}{Simplify the LHS, i.e., if given $\{X\}\rightarrow \{Y\}$ and $\{XB\}\rightarrow \{Y\}$, then we \underline{omit} $\{XB\}\rightarrow \{Y\}$}	
	\end{block}
	$\{C,D\} \rightarrow \{E\},$\\
	$\cancel{\{A,B\} \rightarrow \{B\}}$,\\
	$\cancel{\{A,C,D\} \rightarrow \{E\}}$,\\
	$\{A\}  \rightarrow \{E\},$\\
	$\{D,E\} \rightarrow \{B\}.$\\
	$\{D,E\} \rightarrow \{C\}.$\\
	$\{A\} \rightarrow \{A\}$\\\vspace{5pt}
	\column{0.28\textwidth}  \textbf{STEP 3 (compact)}\\	
	\begin{block}{}
		\tiny \textcolor{red}{Check if every FD is non-trivial and minimal, i.e., if given $\{X\}\rightarrow \{Y\}$, we need to \underline{ensure} there does not exist an $A$ s.t. $\{X\}\rightarrow \{A\}$ and $\{A\}\rightarrow \{Y\}$}	
	\end{block}
	The finalized set is:\\
	$\{C,D\} \rightarrow \{E\},$\\
	$\{A\}  \rightarrow \{E\},$\\
	$\{D,E\} \rightarrow \{B, C\}$\\
	$\cancel{\{A\} \rightarrow \{A\}}$.
	\end{columns}
	
\end{frame}


\begin{frame}[fragile]{\boss{Extra} - Case 6 (Solution)}
	(3)  It is not in 2NF, therefore it is not in 3NF, and not in BCNF, too.\\\vspace{3pt}
	
	E.g., $\{A\} \rightarrow \{E\}$ is not trivial, the LHS is a subset of the candidate key $\{A,D\}$ and the RHS is not a prime attribute, therefore it violates 2NF.\\\vspace{3pt}
	For 3NF and beyond, we find all FDs in the compact minimal cover are violations (all RHS are not prime attributes and all LHS are not superkeys).\\\vspace{5pt}
	
	(4) Recall the compact minimal cover:\\\vspace{3pt}
	
	$\{C,D\} \rightarrow \{E\},$
	$\{A\}  \rightarrow \{E\},$
	$\{D,E\} \rightarrow \{B,C\}.$\\\vspace{3pt}
	
	Let's do 3NF synthesis:\\ \vspace{3pt}
	$R_1 = (\underline{C,D},E)$ (subsumed by $R_3$),
	$R_2 = (\underline{A}, E),$
	$R_3 = (B, C, \underline{D, E}).$\\\vspace{3pt}
	(The candidate key is not included in any relation fragment above, so we add a new fragment manually)\\\vspace{3pt}
	$R_4 = (\underline{A, D}).$\\\vspace{3pt}
	
	The results are $R_2$, $R_3$ and $R_4$.
	

\end{frame}

\begin{frame}[fragile]{\boss{Extra} - Case 6 (Solution)}
	(5) Recall the compact minimal cover:\\
	
	$\{C,D\} \rightarrow \{E\},$
	$\{A\}  \rightarrow \{E\},$
	$\{D,E\} \rightarrow \{B,C\}.$\\\vspace{2pt}
	
	Let's start decomposition at $\{A\} \rightarrow \{E\}$:\\\vspace{2pt}
		
	$R_1 = (\underline{A}, E),$
	We can find that 
	$\Sigma_1 = \{\{A\}\rightarrow\{E\}\},$ and\\\vspace{2pt}
	
	$R_2 = (\underline{A,D}, B, C).$ \\
	{\scriptsize\textcolor{red}{Now it's the most challenging part: \textbf{\underline{find all projected FDs in the fragment}}. Any mistake in this step would affect your determining if the fragment is in BCNF (according to the new candidate key(s)), and consequentially mislead your further decomposition.}}\\
	
	\begin{alertblock}{Find the projected FDs in a fragment}
	Here we find none of FDs in the minimal cover could be preserved in $R_2$. But does it means we lost all FDs? How to find all remaining FDs in fragments?\\
	\end{alertblock}

\end{frame}

\begin{frame}[fragile]{\boss{Extra} - Case 6 (Solution)}
	
	Although $E$ is not presented in $R_2$, we still have $\{A,D\}\rightarrow\{B,C\}$ because of $\{A\}\rightarrow\{E\}$ and $\{D,E\}\rightarrow\{B,C\}$.\\
	One way to find those FDs is to go back and look at the attribute closure, by removing those attributes do not exist in the current fragment, as follows:
	\vspace{-4pt}
	\begin{columns}
	\column{0.45\textwidth}
	\begin{scriptsize}\begin{align*} 
			\Sigma^{+} = \{&\{A\}^{+} \rightarrow \{A,\cancel{E}\},\\
			&\{B\}^{+} \rightarrow \{B\},\\
			&\{C\}^{+} \rightarrow \{C\},\\
			&\{D\}^{+} \rightarrow \{D\},\\
			&\cancel{\{E\}^{+} \rightarrow \{E\}},\\
			&\{A,B\}^{+} \rightarrow \{A,B,\cancel{E}\},\\
			&\{A,C\}^{+} \rightarrow \{A,C,\cancel{E}\},\\
			&\{A,D\}^{+} \rightarrow \{A,B,C,D,\cancel{E}\},\\
			&\cancel{\{A,E\}^{+} \rightarrow \{A,B,E\}\}}\\
			& ...\\
			&\{A,C,D,\cancel{E}\}^{+} \rightarrow \{A,B,C,D,\cancel{E}\}\}\\
			&\{A,B,C,D,\cancel{E}\}^{+} \rightarrow \{A,B,C,D,\cancel{E}\}\}.
	\end{align*}\end{scriptsize} 
	\column{0.53\textwidth}
	Then we know $\{A,D\}$ is the candidate key for $R_2$.\\\vspace{3pt}
	So the first round of BCNF decomposition gives us 2 fragments:\\\vspace{3pt} $R_1=\{\underline{A},E\}$ with $\Sigma_1=\{\{A\}\rightarrow\{E\}\}$;\\ and $R_2=\{\underline{A,D},B,C\}$ with $\Sigma_2=\{\{A,D\}\rightarrow\{B,C\}\}$.\\\vspace{3pt}
	We find both fragment are in BCNF, however we loss all FDs except $\{A\}\rightarrow\{E\}$ from the minimal cover set.\\ \vspace{2pt}
	\end{columns}
	
		
\end{frame}

\begin{frame}[fragile]{\boss{Extra} - Case 6 (Solution)}
	
	If you try to use a FD other than $\{A\}  \rightarrow \{E\}$ in BCNF decomposition, for example $\{C,D\}\rightarrow\{E\}$, you may have two fragments $R_1=\{B,C,D,E\}$ (candidate keys to be unveiled later) and $R_2=\{\underline{A,D},C\}$. \\\vspace{2pt}
	Finding candidate key(s) and minimal cover of $R_1$ is the next challenge:\\\vspace{-3pt}
	\begin{columns}
		\column{0.47\textwidth}
		\begin{scriptsize}\begin{align*} 
				\Sigma_{1}^{+} = \{ & ...\\
				&\{B,E\}^{+} \rightarrow \{B,E\},\\
				&\{C,D\}^{+} \rightarrow \{B,C,D,E\},\\
				&\{C,E\}^{+} \rightarrow \{C,E\},\\
				&\{D,E\}^{+} \rightarrow \{B,C,D,E\},\\
				& ...\\
				&\{C,D,E\}^{+} \rightarrow \{B,C,D,E\},\\
				&\{B,C,D,E\}^{+} \rightarrow \{B,C,D,E\}\}.
		\end{align*}\end{scriptsize} 
		\column{0.48\textwidth}
		With the LHS, we identify $R_1$'s candidate key(s) as $\{D,E\}$ and $\{C,D\}$, and the FDs in the minimal cover of that fragment is $\{\{C,D\}\rightarrow\{E\}, \{D,E\}\rightarrow\{B,C\}\}$. \\\vspace{2pt}
		Both of these two FDs are in BCNF.
	\end{columns}\vspace{-16pt}
	\begin{block}{}
	{\scriptsize According to the textbook referred in this module (Database Management Systems, Ramakrishnan \& Gehrke), there is another (more) ineffective algorithm to decompose $\{A,B\}$ into $\{C,D,E\}$. You can borrow the book from NUS library if you have interest to read.}
	\end{block}
	
\end{frame}

\begin{frame}{}
	\centering  
	For any further question, please feel free to email me:\vspace{10pt}
	
	huasong.meng@u.nus.edu\\\vspace{3pt}
	
	\begin{tcolorbox}
		\begin{center}
			\textcolor{red}{Copyright 2021 Mark H. Meng. All rights reserved.}
		\end{center}
	\end{tcolorbox}
\end{frame}