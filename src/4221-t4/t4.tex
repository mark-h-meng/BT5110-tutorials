\title{CS4221/CS5421}

\subtitle{Tutorial 4: Normal Forms and Normalisation}

\author{Mark Meng Huasong}

\institute[National University of Singapore] % (optional, but mostly needed)
{
	School of Computing\\
	National University of Singapore
}

\titlegraphic{
	\includegraphics[width=2cm]{nus-logo}
}

\date{Week 6, 2022 Spring}

\begin{frame}
	\titlepage
	\begin{tcolorbox}
		\begin{center}
			{\scriptsize \textcolor{red}{All the materials within presentation slides are protected by copyrights.\\
					It is forbidden by NUS to upload these materials to the Internet.}}
		\end{center}
	\end{tcolorbox}
\end{frame}

\begin{comment}
\begin{frame}
	Updated on 9 Oct (Saturday):
	\begin{itemize}
		\item Fixed a FD missing in $R_1$ of Q4(e) in page 29.\\
		i.e., replaced $\Sigma_1 = \{\{A\} \rightarrow \{C\}, \{B\} \rightarrow \{A\}, \{C\} \rightarrow \{D\}\}$ with $\Sigma_1 = \{\{A\} \rightarrow \{C\}, \{A\} \rightarrow \{B\},\{B\} \rightarrow \{A\}, \{C\} \rightarrow \{D\}\}$.
	\end{itemize}
\end{frame}
\end{comment}

\begin{frame}[fragile]{Question 1(a) Preliminary}

Consider the relations $R=\{A, B, C, D, E\}$ with the set of functional dependencies $\Sigma=\{\{A\} \rightarrow \{A, B, C\}, \{A, B\} \rightarrow \{A\}, \{B, C\} \rightarrow \{A, D\}, \{B\} \rightarrow \{A, B\}, \{C\} \rightarrow \{D\}\}$.\\\vspace{5pt}

(a) What preliminary work is needed to study the normalization of $R$ with $\Sigma$? \vspace{15pt}

\textbf{Solution}: Attribute closure, candidate keys \& prime attributes, and compact minimal cover\vspace{5pt}

\textbf{Explanation}: To determine the \textbf{norm form}, you need to know which attribute(s) is/are \textit{\textbf{prime}}. You also need to identify \textbf{\textit{superkeys}}, too. Therefore you need to compute the \textbf{\textit{attribute closure}} at first.\\\vspace{5pt}

To perform \textbf{BCNF decomposition} or \textbf{3NF synthesise} (definitely you'll be asked to do that), you need to have a \textbf{\textit{compact minimal cover}} on hands.

\end{frame}

\begin{frame}[fragile]{Question 1 (a) Cont.}
Let's compute the attribute closures of the attributes in $R$ with $\Sigma$ in order to find the candidate keys of $R$ with $\Sigma$. \vspace{15pt}

The original question is:\\ $\Sigma=\{\{A\} \rightarrow \{A, B, C\}, \{A, B\} \rightarrow \{A\}, \{B, C\} \rightarrow \{A, D\}, \{B\} \rightarrow \{A, B\}, \{C\} \rightarrow \{D\}\}$.\\\vspace{5pt} 

\begin{columns}[t]
	\column{0.45\textwidth}
	\textcolor{gray}{\scriptsize \textit{(Let's start with single attribute first)}}\\
	$\{A\}^{+}= \{A, B, C, D\}$\\	
	$\{B\}^{+}= \{A, B, C, D\}$\\	
	$\{C\}^{+}= \{C, D\}$\\
	$\{D\}^{+}= \{D\}$\\
	$\{E\}^{+}= \{E\}$\\ \vspace{5pt}
	\textcolor{gray}{\textit{\scriptsize (Two attributes' combination)}}\\
	$\{A, B\}^{+}= \{A, B, C, D\}$\\
	$\{A, C\}^{+}= \{A, B, C, D\}$\\
	$\{A, D\}^{+}= \{A, B, C, D\}$
	\column{0.45\textwidth}	
	$\underline{\{A, E\}^{+}}= \{A, B, C, D, E\}$\\
	$\{B, C\}^{+}= \{A, B, C, D\}$\\
	$\{B, D\}^{+}= \{A, B, C, D\}$\\
	$\underline{\{B, E\}^{+}}= \{A, B, C, D, E\}$\\
	$\{C, D\}^{+}= \{C, D\}$\\
	$\{C, E\}^{+}= \{C, D, E\}$\\
	$\{D, E\}^{+}= \{D, E\}$\\ \vspace{5pt}
	Other attribute closures need not be computed.
\end{columns}
\end{frame}

\begin{frame}[fragile]{Question 1 (a) Cont.}
	The candidate keys are $\{A, E\}$ and $\{B, E\}$. \\\vspace{10pt}
	The prime attributes are $A$, $B$ and $E$.\\ \vspace{10pt}
	Superkeys include  $\{A, E\}$, $\{B, E\}$, $\{A, B, E\}$, $\{A, C, E\}$, $\{B, C, E\}$ , $\{A, D, E\}$, $\{B, D, E\}$, $\{A, B, C, E\}$, $\{A, C, D, E\}$, $\{A, B, D, E\}$, $\{B, C, D, E\}$ and $\{A, B, C, D, E\}$.\\ \vspace{15pt}
\end{frame}

\begin{frame}[fragile]{Question 1 (a) Cont.}
	Now let's compute a minimal cover of $R$ with $\Sigma$.\vspace{10pt}
	
	We start from $\Sigma$: \\\vspace{5pt}
	$\{A\} \rightarrow \{A, B, C\}$\\
	$\{A, B\} \rightarrow \{A\}$\\
	$\{B, C\} \rightarrow \{A, D\}$\\
	$\{B\} \rightarrow \{A, B\}$\\
	$\{C\} \rightarrow \{D\}$\\\vspace{5pt}
	Step 1, we simplify the right-hand sides:\\\vspace{3pt}
	\begin{columns}[t]
	\column{0.45\textwidth}
	$\{A\} \rightarrow \{A\}$\\
	$\{A\} \rightarrow \{B\}$\\
	$\{A\} \rightarrow \{C\}$\\
	$\{A, B\} \rightarrow \{A\}$\\
	$\{B, C\} \rightarrow \{A\}$
	\column{0.45\textwidth}
	$\{B, C\} \rightarrow \{D\}$\\
	$\{B\} \rightarrow \{A\}$\\
	$\{B\} \rightarrow \{B\}$\\
	$\{C\} \rightarrow \{D\}$
	\end{columns}
\end{frame}

\begin{frame}[fragile]{Question 1 (a) Cont.}
	Step 2, we simplify the left-hand sides:\\\vspace{3pt}
	$\{A\} \rightarrow \{A\}$.\\	
	$\{A\} \rightarrow \{B\}$.\\	
	$\{A\} \rightarrow \{C\}$.\\	
	$\{A,\cancel{B}\} \rightarrow \{A\}$ because $\{A\} \rightarrow \{A\}$.\\	
	$\{B, \cancel{C}\} \rightarrow \{A\}$ because $\{B\} \rightarrow \{A\}$.\\	
	$\{B, \cancel{C}\} \rightarrow \{D\}$ because $\{B\} \rightarrow \{D\}$ \\\vspace{3pt}
	(we could also remove $\{\cancel{B}, C\} \rightarrow \{D\}$ because $\{C\} \rightarrow \{D\}$). Note that we know that $\{B\} \rightarrow \{D\}$ because $\{B\}^{+}= \{A, B, C, D\}$.\\\vspace{3pt}
	$\{B\} \rightarrow \{A\}$.\\
	$\{B\} \rightarrow \{B\}$.\\
	$\{C\} \rightarrow \{D\}$.
\end{frame}

\begin{frame}[fragile]{Question 1 (a) Cont.}
	Step 3, we simplify the set:\\\vspace{3pt}
	\textit{(for each \{A\}->\{B\}, ensure there does not exist an X that \{A\}->\{X\} and \{X\}->\{B\})}\\\vspace{5pt}
	$\cancel{\{A\} \rightarrow \{A\}}$ because it is trivial.\\	
	$\{A\} \rightarrow \{B\}$.\\	
	$\{A\} \rightarrow \{C\}$.\\	
	$\{B\} \rightarrow \{A\}$.\\	
	$\cancel{\{B\} \rightarrow \{D\}}$ because it can be derived from the others.\\	
	$\cancel{\{B\} \rightarrow \{B\}}$ because it is trivial..\\	
	$\{C\} \rightarrow \{D\}$.\\\vspace{5pt}
	
	\textbf{The result is:} \\\vspace{3pt}
	$\{A\} \rightarrow \{B\}$\\	
	$\{A\} \rightarrow \{C\}$\\	
	$\{B\} \rightarrow \{A\}$\\
	$\{C\} \rightarrow \{D\}$
\end{frame}

\begin{frame}[fragile]{Question 1 (a) Cont.}
	\begin{alertblock}{Notice}
	Note that there can be other minimal covers that the algorithm can compute by considering the functional dependencies in a different order at each step of the algorithm. This is not the case in the example.
	\end{alertblock}\vspace{5pt}

	However, there is a minimal cover that the algorithm cannot compute:\\\vspace{5pt}
	$\{A\} \rightarrow \{B\}$\\
	$\{B\} \rightarrow \{A\}$\\
	$\{B\} \rightarrow \{C\}$\\
	$\{C\} \rightarrow \{D\}$\\\vspace{10pt}	
	If the algorithm starts from $\Sigma^{+}$, then it can find all minimal covers.
\end{frame}

\begin{frame}[fragile]{Question 1 (a) Cont.}
	Next, let's compute a compact minimal cover of $R$ with $\Sigma$.\vspace{10pt}
	
	Given the minimal cover of $R$ with $\Sigma$:\\\vspace{3pt}
	$\{A\} \rightarrow \{B\}$\\	
	$\{A\} \rightarrow \{C\}$\\	
	$\{B\} \rightarrow \{A\}$\\
	$\{C\} \rightarrow \{D\}$\\\vspace{5pt}
	The \textbf{compact minimal cover} is:\\\vspace{3pt}
	$\{A\} \rightarrow \{B, C\}$\\	
	$\{B\} \rightarrow \{A\}$\\	
	$\{C\} \rightarrow \{D\}$
\end{frame}

\begin{frame}[fragile]{Question 1 (a) Cont.}
Now let's look back the relation $R$ with all FDs ($\Sigma$) \textcolor{red}{\textbf{with this un-official visualization}}:\\\vspace{5pt}
\begin{figure}
	\includegraphics[width=0.85\textwidth, trim=0 0 0 0, clip]{t5/images/end_q1.png}
\end{figure}

\textit{(We will keep using this (kind of) figures for later demonstrations)}
\end{frame}

\begin{frame}[fragile]{Question 1 (b-c) Normal Forms}
(*) Is $R$ with $\Sigma$ in 2NF? (Just for recap purpose)\\\vspace{10pt}
(b) Is $R$ with $\Sigma$ in 3NF?\\\vspace{10pt}
(c) Is $R$ with $\Sigma$ in BCNF?\\\vspace{10pt}
\end{frame}

\begin{frame}[fragile]{Question 1 (b-c) Cont.}
(*) Is $R$ with $\Sigma$ in 2NF?\\\vspace{10pt}

\textbf{Recap}: How to \textcolor{brown}{mechanically} define 2NF?\\\vspace{10pt}
A relation $R$ with a set of functional dependencies $\Sigma$ is in second norm form (2NF) if and only iff for every functional dependency $X\rightarrow\{A\}$, we have:
\begin{itemize}
	\item $X\rightarrow\{A\}$ is trivial, or
	\item $X$ is not a subset of a candidate key, or
	\item $A$ is a prime attribute.
\end{itemize}\vspace{10pt}

\begin{alertblock}{Just to understand, no need to keep it in mind!}
	Unlike the 1NF that allow any attribute as long as they are single values, 
	The 2NF is more strict so that it does not allow any non-prime attribute to be determined by partial (not full) candidate keys.
\end{alertblock}

\end{frame}

\begin{frame}[fragile]{Question 1 (b-c) Cont.}
\begin{figure}
	\includegraphics[width=0.85\textwidth, trim=0 0 0 0, clip]{t5/images/q3_2nf_highlight.png}
\end{figure}
\end{frame}

\begin{frame}[fragile]{Question 1 (b-c) Cont.}
\textbf{Solution}:
Let us look at the non-trivial functional dependencies of the form $X \rightarrow \{A\}$ derived from $\Sigma$. Namely after removing the trivial functional dependencies after step 1 of the minimal cover algorithm. Equivalently, we could use a minimal cover.\\\vspace{3pt}
$\{A\} \rightarrow \{C\}$ is non-trivial, $\{A\}$ is a proper subset of a candidate key and $\{C\}$ is not a prime attribute. This functional dependency violates the three conditions of the 2NF definition. $R$ with $\Sigma$ is not in 2NF.\\\vspace{3pt}

Incidentally, several other functional dependencies also violate the 2NF definition:\\
$\{B\} \rightarrow \{C\}$ is non-trivial, $\{B\}$ is a proper subset of a candidate key and $\{C\}$ is not a prime attribute. \\\vspace{3pt}
This ones do not (one condition is met):\\
$\{B,C\} \rightarrow \{D\}$ $\{B,C\}$ is not a proper subset of a candidate key.\\
$\{C\} \rightarrow \{D\}$ $\{C\}$ is not a proper subset of a candidate key.\\
$\{A\} \rightarrow \{B\}$ $\{B\}$ is a prime attribute.\\
$\{B, C\} \rightarrow \{A\}$ $\{A\}$ is a prime attribute.
\end{frame}

\begin{frame}[fragile]{Question 1 (b-c) Cont.}

(b) Is $R$ with $\Sigma$ in 3NF?\\\vspace{10pt}

\textbf{Recap}: How to \textcolor{brown}{mechanically} define 3NF?\\\vspace{10pt}
A relation $R$ with a set of functional dependencies $\Sigma$ is in third norm form (3NF) if and only iff for every functional dependency $X\rightarrow\{A\}$, we have:
\begin{itemize}
	\item $X\rightarrow\{A\}$ is trivial, or
	\item $X$ is a superkey, or
	\item $A$ is a prime attribute.
\end{itemize}\vspace{5pt}

\begin{alertblock}{Just to understand, no need to keep it in mind!}
	Unlike the 2NF that still allows a non-prime attribute to be determined by another non-prime attribute.\\ 
	The 3NF is more strict so that it does not allow any non-prime attribute to be determined by other non-prime attributes (all non-prime attributes have to be independent to each other, and therefore can only be determined by prime attributes).
\end{alertblock}

\end{frame}

\begin{frame}[fragile]{Question 1 (b-c) Cont.}
\begin{figure}
	\includegraphics[width=0.95\textwidth, trim=0 0 0 0, clip]{t5/images/q3_3nf_highlight.png}
\end{figure}
\end{frame}

\begin{frame}[fragile]{Question 1 (b-c) Cont.}
\textbf{Solution}:
Let us look at the non-trivial functional dependencies of the form $X \rightarrow \{A\}$ derived from $\Sigma$. Namely after removing the trivial functional dependencies after step 1 of the minimal cover algorithm. Equivalently, we could use a minimal cover.\\\vspace{3pt}
$\{A\} \rightarrow \{C\}$ is non-trivial, $\{A\}$ is not a superkey and $\{C\}$ is not a prime attribute. This functional dependency violates the three conditions of the 3NF definition. $R$ with $\Sigma$ is not in 3NF.\\\vspace{3pt}

\end{frame}

\begin{frame}[fragile]{Question 1 (b-c) Cont.}
Incidentally, several other functional dependencies also violate the 3NF definition:\\

$\{B, C\} \rightarrow \{D\}$ is non-trivial, $\{B, C\}$ is not a superkey and $\{D\}$ is not a prime attribute.\\
$\{B\} \rightarrow \{C\}$ is non-trivial, $\{B\}$ is not a superkey and $\{C\}$ is not a prime attribute.\\
$\{C\} \rightarrow \{D\}$ is non-trivial, $\{C\}$ is not a superkey and $\{D\}$ is not a prime attribute.\\\vspace{10pt}

This two do not (one condition is met):\\
$\{A\} \rightarrow \{B\}$ is non-trivial, $\{A\}$ is not a superkey but $\{B\}$ is a prime attribute.\\\vspace{3pt}
$\{B, C\} \rightarrow \{A\}$ $\{A\}$ is a prime attribute.\\\vspace{3pt}

\end{frame}

\begin{frame}[fragile]{Question 1 (b-c) Cont.}
(c) Is $R$ with $\Sigma$ in BCNF?\\\vspace{10pt}

\textbf{Recap}: How to \textcolor{brown}{mechanically} define BCNF?\\\vspace{10pt}
A relation $R$ with a set of functional dependencies $\Sigma$ is in Boyce-Codd norm form (BCNF) if and only iff for every functional dependency $X\rightarrow\{A\}$, we have:
\begin{itemize}
	\item $X\rightarrow\{A\}$ is trivial, or
	\item $X$ is a superkey.
\end{itemize}\vspace{3pt}

\begin{alertblock}{Just to understand, no need to keep it in mind!}
BCNF was proposed in 1974 (3 years later than 2NF and 3NF) and was known as 3.5NF at the beginning.\\\vspace{3pt}
Unlike the 3NF that does not allow a non-prime attribute to be determined by other non-prime attributes (e.g. \texttt{city} is determined by \texttt{postal} in a database that record residence details). 
The BCNF is more strict so that it only allow attribute(s) to be determined by superkeys.
\end{alertblock}

\end{frame}

\begin{frame}[fragile]{Question 1 (b-c) Cont.}
\begin{figure}
	\includegraphics[width=0.85\textwidth, trim=0 0 0 0, clip]{t5/images/q3_bcnf_highlight.png}
\end{figure}
\end{frame}

\begin{frame}[fragile]{Question 1 (b-c) Cont.}
\textbf{Solution}:
$R$ with $\Sigma$ is not in 3NF, then it cannot be in BCNF.\\\vspace{3pt}

Let us look at the non-trivial functional dependencies of the form $X \rightarrow \{A\}$ derived from $\Sigma$. Namely after removing the trivial functional dependencies after step 1 of the minimal cover algorithm. Equivalently, we could use a minimal cover.\\\vspace{3pt}

$\{A\} \rightarrow \{C\}$ is non-trivial and $\{A\}$ is not a superkey . This functional dependency violates the two conditions of the BCNF definition. $R$ with $\Sigma$ is not in BCNF.\\\vspace{3pt}

Incidentally, all the other functional dependencies also violate the BCNF definition:\\
$\{A\} \rightarrow \{B\}$ is non-trivial and $\{A\}$ is not a superkey.\\
$\{B, C\} \rightarrow \{D\}$ is non-trivial and $\{B, C\}$ is not a superkey.\\
$\{B\} \rightarrow \{C\}$ is non-trivial and $\{B\}$ is not a superkey.\\
$\{C\} \rightarrow \{D\}$ is non-trivial and  $\{C\}$ is not a superkey.\\
$\{B, C\} \rightarrow \{A\}$ $\{A\}$ is non-trivial and  $\{B, C\}$ is not a superkey.
\end{frame}

\begin{frame}[fragile]{Question 1 (d-j) Normalisation}
	(d) Decompose $R$ with $\Sigma$ into a BCNF decomposition using the algorithm from the lecture.\\\vspace{5pt}
	
	\textbf{Solution}: We found that $\{A\} \rightarrow \{C\}$ violates the BCNF condition.\\\vspace{3pt}
	
	We use it to decompose into two fragments:\\
	$R_1 = \{A\}^{+} = \{A, B, C, D\}$ with $\Sigma_1 = \{\{A\} \rightarrow \{C\}, \{A\} \rightarrow \{B\},\{B\} \rightarrow \{A\}, \{C\} \rightarrow \{D\}\}$.
	
	$R_2 = \{A, E\}$ with $\Sigma_2 = \emptyset$.
	
	$R_2$ with $\Sigma_2$ is in BCNF.
	
	$R_1$ with $\Sigma_1$ is not in BCNF because $\{C\} \rightarrow \{D\}$ violates the BCNF condition.
	\textbf{$\leftarrow$ We need to further decompose it.}
\end{frame}

\begin{frame}[fragile]{Question 1 (d-j) Cont.}
	For $R_1 = \{A\}^{+} = \{A, B, C, D\}$, we use it to decompose into two fragments:\\\vspace{5pt}
	$R_{1.1} = \{C\}^{+} = \{C, D\}$ (computed for $R_1$ with $\Sigma_1$!) with $\Sigma_{1.1} = \{\{C\} \rightarrow \{D\}\}$.
	
	$R_{1.2} = \{A, B, C\}$ with $\Sigma_{1.2} = \{\{A\} \rightarrow \{B\}, \{A\} \rightarrow \{C\}, \{B\} \rightarrow \{A\}\}$.\\\vspace{5pt}
	
	$R_{1.1}$ with $\Sigma_{1.1}$ is in BCNF.
	
	$R_{1.2}$ with $\Sigma_{1.2}$ is in BCNF.\\\vspace{5pt}
	
	The result is a decomposition into three fragments: $R_2$, $R_{1.1}$ and $R_{1.2}$.\\\vspace{5pt}
\end{frame}

\begin{frame}[fragile]{Question 1 (d-j) Cont.}
	(e) Is the results lossless?\\\vspace{5pt}
	\textbf{Solution}: Yes, the algorithm guarantees that the result is lossless.\\
	\vspace{30pt}
	(f) Is the results dependency preserving?\\\vspace{5pt}
	\textbf{Solution}:
	Incidentally yes, the \textbf{algorithm} (itself) \textcolor{red}{\textbf{does not}} guarantee that the result is dependency preserving.\\
	\begin{alertblock}{Notice}
		Don't take the dependency preserving for granted in decomposition.
	\end{alertblock}
	
\end{frame}

\begin{frame}[fragile]{Question 1 (d-j) Cont.}
(g) Synthesise $R$ with $\Sigma$ into a 3NF decomposition using the algorithm from the lecture.\\\vspace{5pt}

\textbf{Solution}:
First compute a compact minimal cover:

$\{A\} \rightarrow \{B, C\}$

$\{B\} \rightarrow \{A\}$

$\{C\} \rightarrow \{D\}$

For each functional dependency create a fragment:

$R_1 = \{A, B, C\}$

$R_2 = \{B, A\}$ , this fragment is not kept: it is subsumed by $R_1$.

$R_3 = \{C, D\}$.

None of the fragments contain a candidate key. We choose one: say $\{A, E\}$, and add it as fragment:

$R_4 = \{A, E\}$.

The result is:
$R_1 = \{\underline{A}, \underline{B}, C\}$
$R_3 = \{\underline{C}, D\}$.
$R_4 = \{\underline{A, E}\}$.

The algorithm always works. It is guaranteed to find a decomposition in 3NF. Note that using a different minimal cover or compact minimal cover may give a different (but equally correct) result.
\end{frame}

\begin{frame}[fragile]{Question 1 (d-j) Cont.}
(h) Is the results lossless?\\\vspace{5pt}
\textbf{Solution}: Yes, the algorithm guarantees that the result is lossless.
\\\vspace{30pt}
(i) Is the results dependency preserving?\\\vspace{5pt}
\textbf{Solution}:
Yes, the algorithm guarantees that the result is dependency preserving.

\end{frame}

\begin{frame}[fragile]{Question 1 (d-j) Cont.}
(j) Is the results in BCNF?\\\vspace{5pt}

\textbf{Solution}:
The algorithm guarantees that the result is in 3NF. In this case, it is also in BCNF! (you can check). It is not always but often the case.\\\vspace{10pt}

\textcolor{red}{\textbf{Extra}}:\\
To wrap-up, BCNF decomposition is \textbf{not always optimal}, as it aims to dismiss some ``unnecessary'' dependencies among attributes. (sometimes 3NF is good enough)\\\vspace{5pt}
\end{frame}


\begin{frame}{}
	\begin{figure}
		\includegraphics[width=0.6\textwidth, trim=0 3.5cm 0 0, clip]{t5/images/final.png}
	\end{figure}
\begin{center}
	Enjoy the recess week and have a good rest. Keep safe!
\end{center} 
\end{frame}

\begin{frame}{}
	\centering  
	For any further question, please feel free to email me:\vspace{10pt}
	
	huasong.meng@u.nus.edu\\\vspace{3pt}
	%menghs@i2r.a-star.edu.sg (work)\\\vspace{3pt}
	%huasong.meng@gmail.com (personal)\vspace{10pt}
	
	Or you can whatsapp/wechat me via: 81028639 \vspace{20pt}
	
	\begin{tcolorbox}
		\begin{center}
			\textcolor{red}{Copyright 2021 Mark H. Meng. All rights reserved.}
		\end{center}
	\end{tcolorbox}
\end{frame}